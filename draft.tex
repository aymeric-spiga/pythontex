\documentclass[12pt,hyperref]{article}
\usepackage[utf8]{inputenc}
\usepackage[english]{babel}
\usepackage{graphicx}
\usepackage{natbib}
\usepackage{color}
\usepackage{amsmath}
\usepackage[margin=1.5cm]{geometry}
\usepackage{pythontex}

%%%%%%%%%%%%%%%%%%%%%%%%%%
%%% PYTHONTEX SETTINGS %%%
%%%%%%%%%%%%%%%%%%%%%%%%%%
\setpythontexworkingdir{.}
%%
\begin{pycode}
import sys
import os
def add_to_path(p):
    if p and p not in sys.path:
        sys.path.append(p)
\end{pycode}
%%
\newcommand{\importandtypeset}[2][.]{%
  \pyc{add_to_path('#1'); from #2 import *; pytex.add_dependencies(os.path.join('#1', '#2.py'))}%
  \inputpygments{python}{#1/#2.py}%
}
%%

%%%%%%%%%%%%%%%%%%%%%%%%%%%%%
%%% PYTHON GENERAL IMPORT %%%
%%%%%%%%%%%%%%%%%%%%%%%%%%%%%
\begin{pythontexcustomcode}{py}
# IMPORTS
import numpy   as np
import scipy   as sp
import sympy   as sym
import planets as pln
import pylab   as plt
from mpl_toolkits.basemap import Basemap
# SETTINGS 
from matplotlib import rc
rc('text', usetex=True)
rc('font', family='serif', serif='Times', size=12)
# FUNCTIONS
# printeg( [char1,char2,...] )
# -- prints char1 = char2 = ...
def printeg(arg,symbol=' = ',box=False):
  zechar = '\[ '
  if box: zechar = zechar + ' \\boxed{ '
  nnn = len(arg)
  for ii in range(nnn):
    zechar = zechar + sym.latex(arg[ii])
    if ii != nnn-1:
      zechar = zechar + symbol
  if box: zechar = zechar + ' } '  
  zechar = zechar + ' \]'
  print(zechar)
\end{pythontexcustomcode}
%%%%%%%%%%%%%%%%%%%%%%%%%%%%%%%%%%%%%%%%%%%%%%%%%%%%%%%%%%%%%%


\title{Cours}
\author{Aymeric Spiga}



\begin{document}
\maketitle
%\tableofcontents






\section{Flux incident}

\subsection{Variation géographique du flux incident (Loi de Malus)}

\begin{pyblock}[incflux][fontsize=\scriptsize]
E, Eloc, R = sym.symbols('\mathcal{E} \mathcal{E}_{loc} R', positive=True)
phi, lam = sym.symbols('\phi \lambda')
malus = E*sym.cos(phi)*sym.cos(lam)
printeg( [Eloc,malus] )
\end{pyblock} 
\printpythontex

\begin{pyblock}[incflux][fontsize=\scriptsize]
Elocnumf = sym.lambdify((lam, phi), malus.subs(E,1), "numpy")
angles = np.linspace(-np.pi/2,np.pi/2,num=25)
X,Y = np.meshgrid(angles, angles)
m = Basemap(projection='ortho',lon_0=20,lat_0=20,resolution='l')
m.drawmapboundary(fill_color='k')
mx, my = m(180.*X/np.pi,180.*Y/np.pi)
m.contourf(mx,my,Elocnumf(X,Y),cmap=plt.cm.hot,levels=np.linspace(0,1,15))
m.drawcoastlines()
mx, my = m(0,0) 
plt.plot(mx,my,'kx')
plt.savefig('eloc.pdf', bbox_inches='tight') ; plt.close()
\end{pyblock}
\begin{figure}[h!]
\begin{center}
\includegraphics[width=0.25\textwidth]{eloc.pdf}
\end{center}
\end{figure}

\subsection{Flux incident intégré sur la sphère : le facteur 4}

\begin{pyblock}[incflux][fontsize=\scriptsize]
Pi, Phii = sym.symbols('\mathcal{P}_i,\Phi_i', positive=True)
integrand = Eloc*(R**2)*sym.cos(lam)
i = sym.Integral(integrand,(phi,-sym.pi/2,sym.pi/2),(lam,-sym.pi/2,sym.pi/2))
ii = i.subs(Eloc,malus)
iii = ii.doit()
printeg( [Pi,i,ii,iii] )
sphere = 4*sym.pi*(R**2)
printeg( [Phii, Pi/sphere, iii/sphere], box=True )
\end{pyblock} 
\printpythontex

\section{Flux émis}

\subsection{Loi de Planck}

\begin{pyblock}[planck][fontsize=\scriptsize]
BlT, h, c, lam, k, temp = sym.symbols('B_{\lambda}(T) h c \lambda k_B T') #, commutative=True)
term = (h*c) / (lam*k*temp)
planck = ( 2*h*c**2/(lam**5) ) / ( sym.exp(term) - 1 )
printeg( [BlT,planck] )
\end{pyblock}
\printpythontex

\subsection{Loi de Wien}

\begin{pyblock}[planck][fontsize=\scriptsize]
x,xm,lm = sym.symbols('x x_{max} \lambda_{max}')
zemax = r'\frac{\partial B}{\partial \lambda}=0\qquad\textrm{en posant}\qquad '
print('\['+zemax+sym.latex(sym.Eq(x,term))+'\]')
f = sym.Function('B')(lam)
f = planck # faire de la planckienne une fonction à une variable
df = f.diff(lam)
df = df.subs(lam,term.subs(lam,x)) # nécessaire pour solve
df = sym.simplify(df) # esthétique
i = sym.Eq(df,0)
ii = sym.solve(i,x)[0]
printeg( [i,sym.Eq(xm,ii)], symbol=' \qquad\Rightarrow\qquad ' ) 
\end{pyblock}
\printpythontex

\begin{itemize}
\item On the Lambert W function, Corless et al. (1993)
\item The Lambert W Function and Quantum Statistics, Valluri et al. (2009)
\end{itemize}

\begin{pyblock}[planck][fontsize=\scriptsize]
iii = term.subs(lam,xm)
wien = term.subs([ (lam,ii), (h,pln.h), (c,pln.c), (k,pln.k) ]).evalf(4)
printeg( [lm,iii,wien],  box=True )
\end{pyblock}
\printpythontex

\subsection{Emission thermique typique d'étoiles et planètes}

\begin{pyblock}[planck][fontsize=\scriptsize]
plancknum = planck.subs([ (h,pln.h), (c,pln.c), (k,pln.k) ])
plancknumf = sym.lambdify((lam, temp), plancknum, "numpy")
numlam = np.logspace(-2,2,num=100)
def plancknumfred(temp): return plancknumf(1e-6*numlam,temp)
kel = [6000.,1000.,pln.Venus.T0,pln.Earth.T0,pln.Mars.T0,pln.Jupiter.T0,pln.Pluto.T0]
col = ["y","g","c","b","r","m","k"]
lab = ["Soleil","Naine brune","Venus","Terre","Mars","Jupiter","Pluton"]
for i in range(len(kel)):
  plt.plot(numlam,plancknumfred(kel[i]),color=col[i],label=lab[i])
plt.semilogx()
plt.xlabel("Longueur d'onde ($\mu$m)")
plt.semilogy()
plt.ylabel(r'W~m$^{-2}$~sr$^{-1}$~$\mu$m$^{-1}$')
plt.ylim((1e-1,1e14))
plt.title(sym.latex(BlT,mode="inline"))
plt.legend(loc="lower left")
plt.savefig('planck.pdf', bbox_inches='tight') ; plt.close()
\end{pyblock}
\begin{figure}[h!]
\begin{center}
\includegraphics[width=0.5\textwidth]{planck.pdf}
\end{center}
\end{figure}

\begin{pyblock}[planck][fontsize=\scriptsize]
print(r"\begin{center}\begin{tabular}{|c|c|c|}\hline")
title = r" &" 
title = title + sym.latex(temp,mode='inline') + r"~(K)" + r"&"
title = title + sym.latex(lm,mode='inline') + r"~($\mu$m)" + r"\\ \hline"
print(title) 
for i in range(len(kel)):
  line = r"%s & %.0f & %.1f \\ \hline"
  print(line % (lab[i],kel[i],1.e6*wien.subs(temp,kel[i])))  
print(r"\end{tabular}\end{center}")
\end{pyblock} 
\printpythontex

\section{Equilibre radiatif}

\subsection{Loi de Stefan-Boltzmann}

\begin{pyblock}[planck][fontsize=\scriptsize]
x = sym.Symbol('x')
ib = sym.Integral(BlT,lam)
i = sym.Integral(planck,lam) # integration
f = lambda x: (h*c) / (x*k*temp) # changement de variable
ii = i.transform(lam, (f(x),x)).doit() 
printeg( [ib,i,ii] )
\end{pyblock}
\printpythontex

%%%%%%%%%%%%%
%L'intégrale à droite est un cas particulier de la distribution de Bose-Einstein
%qui se calcule à l'aide du polylogarithme (ou fonction de Jonquière 1889)
%%http://mathworld.wolfram.com/Bose-EinsteinDistribution.html
%The integral on the right is standard and goes by many names: it is a particular case of a Bose-Einstein integral, or the Riemann zeta function zeta(4), or the Polylogarithm.
%A special case of the Bose-Einstein distribution.
%\[ Li_4(1) \Gamma(4) \]
%Jonquière, Bulletin de la S. M. F., tome 17 (1889), p. 142-152
%\begin{pyblock}[planck][fontsize=\scriptsize]
%n,alph = sym.symbols('n \alpha')
%f = x**n / (sym.exp(x)-1)
%i = sym.Integral(f,(x,0,sym.oo)) 
%#ii = i.doit() # ne sait pas faire
%ii = sym.functions.special.zeta_functions.zeta(4,1) * sym.functions.special.gamma_functions.gamma(4)
%#ii = sym.functions.special.zeta_functions.polylog(n+1,alph) * sym.functions.special.gamma_functions.gamma(n+1)
%printeg( [i,ii] )
%\end{pyblock}
%\printpythontex
%%%%%%%%%%%%%

\begin{pyblock}[dlsigmaint][fontsize=\scriptsize]
eps,x = sym.symbols('\epsilon x',positive=True)
g = eps / (1-eps)
dl = sym.series(g,eps,0,5)
printeg( [g,dl] )
printeg( [x**3*g.subs(eps,sym.exp(-x)),x**3*dl.subs(eps,sym.exp(-x))] )
\end{pyblock}
\printpythontex

\begin{pyblock}[sigmaint][fontsize=\scriptsize]
k,u,x = sym.symbols('k u x',positive=True)
f = x**3 / (sym.exp(x)-1)
i = sym.Integral(f,(x,0,sym.oo)) # .doit() marche pas
g = x**3 * sym.exp(-k*x)
s = sym.Sum(g,(k,1,sym.oo)) # .doit() same as above
ii = sym.Integral(s,(x,0,sym.oo)) #
iii = sym.Sum( sym.Integral(g,(x,0,sym.oo)), (k,1,sym.oo))
iv = sym.Sum( sym.laplace_transform(x**3, x, k)[0], (k,1,sym.oo))
v = iv.doit()
printeg( [i,ii,iii,iv,v] )
\end{pyblock}
\printpythontex

\noindent Chaque terme dans la somme est une intégrale convergente, à savoir : une transformée
de Laplace de la fonction cube, donc le terme de sommation peut être sorti de l'intégrale.

how to replace the integral by the result of the previous calculation

\begin{pyblock}[planck][fontsize=\scriptsize]
x = sym.Symbol('x',real=True)
f = x**3 / (sym.exp(x)-1)
cint = sym.Integral(f)
vint = sym.pi**4 / 15
fac = ii.subs(cint,vint)
fac = sym.ratsimp(ii / cint)
fac = ii.subs(x,0)
printeg( [ib,fac] )
\end{pyblock}
\printpythontex

%#eq = sym.Eq(x**3,sym.exp(x)-1)
%#sol = sym.solve(eq)
%#printeg( [sol] ) 

%\begin{pyblock}[sigmaint][fontsize=\scriptsize]
%BlT,lam = sym.symbols('B_{\lambda}(T) \lambda')
%ib = sym.Integral(BlT,(lam,0,sym.oo))
%printeg( [ib,v], box=True )
%\end{pyblock}
%\printpythontex


%\begin{pyblock}[sigmaint][fontsize=\scriptsize]
%alt = ii.transform(x,(u/k,u))
%printeg( [ii,alt] )
%\end{pyblock}
%\printpythontex



\end{document}




\end{document}

\begin{pyblock}[planck][fontsize=\scriptsize]
eps,k,u,n,N = sym.symbols('\epsilon k u n N',positive=True)
## marche pas 
i = sym.Integral((x**3)*sym.exp(-k*x),(x,0,sym.oo))
ii = i.transform(x,(u/k,u))
iii = ii.doit()
printeg( [i,ii,iii] )
intsum = sym.Integral(sss,(x,0,n))
sumint = sym.Sum(iii,(k,1,N))
sumintdo = sumint.doit()
printeg( [intsum,sumint,sumintdo] )
ll = sym.limit(sumintdo, N, sym.oo)
printeg( [intsum.subs(N,sym.oo),ll] )
\end{pyblock}
\printpythontex



%#sympy does not know how to do this
%#https://en.wikipedia.org/w/index.php?oldid=293273084#Appendix
%n = sym.symbols('n')
%i = sym.Integral(x**n/(sym.exp(x)-1),(x,0,np.inf))
%ii = sym.functions.special.zeta_functions.zeta(n+1,1) * sym.functions.special.gamma_functions.gamma(n+1)
%printeg( [i,ii] )
%printeg( [i.subs(n,3),ii.subs(n,3)] )



\end{document}

\subsection{Equilibre radiatif simple}

%\begin{pyblock}[][fontsize=\scriptsize]
%Ab, Ep, Phii, Teq, sig = sym.symbols('A_b \mathcal{E}^{\prime} \Phi_i T_{eq} \sigma', positive=True)
%sphere = 4*sym.pi*(R**2)
%tr = 1-Ab 
%expr = tr*iii.subs(E,4*Ep)/sphere
%printeg( [Phii, tr*i/sphere, tr*iii/sphere, expr] )
%Phii = expr
%\end{pyblock} 
%\printpythontex


%% $\sigma=\py{sym.Float(pln.sigma,3)}$
\begin{pyblock}[][fontsize=\scriptsize]
print(r"\begin{center}\begin{tabular}{|c|c|c|c|c|}\hline")
title = r" &" 
title = title + sym.latex(Ab,mode='inline') + r"&"
title = title + sym.latex(Ep,mode='inline') + r"~(W~m$^{-2}$)" + r"&" 
title = title + sym.latex(Teq,mode='inline') + r"~(K)" + r"&" 
title = title + r"$T_{s}~(K)$" + r"\\ \hline"
print(title) 
myp = pln.Planet()
for ppp in ["Mercury","Venus","Earth","Mars","Jupiter","Saturn","Titan","Pluto"]:
  myp.ini(ppp)
  exprnum = expr.subs([ (Ab,myp.albedo), (Ep,myp.L/4.), (sig,pln.sigma) ]) 
  line = r"%s & %.2f & %.1f & %.0f & %.0f \\ \hline"
  print(line % (myp.name,myp.albedo,myp.L/4.,exprnum,myp.Tsbar))
print(r"\end{tabular}\end{center}")
\end{pyblock} 
\printpythontex


\end{document}






\end{document}




\end{document}


\end{document}




%%%% NON
%\begin{pycode}
%tau,sig,olr = sym.symbols('\tau \sigma OLR')
%fp,fm,T = sym.symbols('F^+ F^- T',cls=sym.Function)
%eq1 = sym.Eq( fp(tau).diff(tau)-fp(tau) , -fm(tau)-fm(tau).diff(tau) )
%ics = {fp(0): olr, fm(0): 0}
%sol = sym.dsolve(eq1,[fp(tau),fm(tau)], ics=ics)
%#printeg( [sys,sol], symbol=' \qquad\Rightarrow\qquad ' )
%\end{pycode}
%\end{document}









\begin{pycode}

i = sym.symbols('i')
x = sym.symbols('x',positive=True)
tef = sym.summation(sym.exp(-i*x),(i,0,np.inf))
print('\[' + sym.latex(tef) + '\]')



#i = sp.integrate(planck,(lam,0,inf))
#print('\[' + latex(i) + '\]')
#i = sp.integrals.integrate(planck,(lam,1e-9,1e3))
#print('\[' + latex(i) + '\]')
## marche pas
#i = Integral(interm,(lam,0,inf))
#ii = i.doit()
#print('\[' + latex(i) + ' = ' + latex(ii) + '\]')

\end{pycode}
\end{document}


\begin{pyblock}[][fontsize=\scriptsize]
expr = sig*teq**4
printeg( [Phio,expr] )
Phio = expr

eqrad = sym.Eq(Phii,Phio)
print(sym.latex(eqrad,mode='equation'))

sol = sym.solve(Phii-Phio,teq**4) 
expr = sym.root(sol[0],4)
print(sym.latex(sym.Eq(teq,expr),mode='equation'))
\end{pyblock} 
\printpythontex


$h=\pyc{planets.h}$



\importandtypeset{escape}
\pyc{pytex.add_created("escape.pdf")}
\includegraphics[width=\textwidth]{escape.pdf}

%%\begin{pyblock}
%\begin{pycode}
%\input{escape.py}
%\end{pycode}
%%\end{pyblock}
%\includegraphics[width=\textwidth]{escape.pdf}


\end{document}
